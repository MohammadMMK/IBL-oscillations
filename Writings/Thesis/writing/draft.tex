% Options for packages loaded elsewhere
\PassOptionsToPackage{unicode}{hyperref}
\PassOptionsToPackage{hyphens}{url}
\PassOptionsToPackage{dvipsnames,svgnames,x11names}{xcolor}
%
\documentclass[
  letterpaper,
  DIV=11,
  numbers=noendperiod]{scrartcl}

\usepackage{amsmath,amssymb}
\usepackage{setspace}
\usepackage{iftex}
\ifPDFTeX
  \usepackage[T1]{fontenc}
  \usepackage[utf8]{inputenc}
  \usepackage{textcomp} % provide euro and other symbols
\else % if luatex or xetex
  \usepackage{unicode-math}
  \defaultfontfeatures{Scale=MatchLowercase}
  \defaultfontfeatures[\rmfamily]{Ligatures=TeX,Scale=1}
\fi
\usepackage{lmodern}
\ifPDFTeX\else  
    % xetex/luatex font selection
\fi
% Use upquote if available, for straight quotes in verbatim environments
\IfFileExists{upquote.sty}{\usepackage{upquote}}{}
\IfFileExists{microtype.sty}{% use microtype if available
  \usepackage[]{microtype}
  \UseMicrotypeSet[protrusion]{basicmath} % disable protrusion for tt fonts
}{}
\makeatletter
\@ifundefined{KOMAClassName}{% if non-KOMA class
  \IfFileExists{parskip.sty}{%
    \usepackage{parskip}
  }{% else
    \setlength{\parindent}{0pt}
    \setlength{\parskip}{6pt plus 2pt minus 1pt}}
}{% if KOMA class
  \KOMAoptions{parskip=half}}
\makeatother
\usepackage{xcolor}
\setlength{\emergencystretch}{3em} % prevent overfull lines
\setcounter{secnumdepth}{-\maxdimen} % remove section numbering
% Make \paragraph and \subparagraph free-standing
\ifx\paragraph\undefined\else
  \let\oldparagraph\paragraph
  \renewcommand{\paragraph}[1]{\oldparagraph{#1}\mbox{}}
\fi
\ifx\subparagraph\undefined\else
  \let\oldsubparagraph\subparagraph
  \renewcommand{\subparagraph}[1]{\oldsubparagraph{#1}\mbox{}}
\fi


\providecommand{\tightlist}{%
  \setlength{\itemsep}{0pt}\setlength{\parskip}{0pt}}\usepackage{longtable,booktabs,array}
\usepackage{calc} % for calculating minipage widths
% Correct order of tables after \paragraph or \subparagraph
\usepackage{etoolbox}
\makeatletter
\patchcmd\longtable{\par}{\if@noskipsec\mbox{}\fi\par}{}{}
\makeatother
% Allow footnotes in longtable head/foot
\IfFileExists{footnotehyper.sty}{\usepackage{footnotehyper}}{\usepackage{footnote}}
\makesavenoteenv{longtable}
\usepackage{graphicx}
\makeatletter
\def\maxwidth{\ifdim\Gin@nat@width>\linewidth\linewidth\else\Gin@nat@width\fi}
\def\maxheight{\ifdim\Gin@nat@height>\textheight\textheight\else\Gin@nat@height\fi}
\makeatother
% Scale images if necessary, so that they will not overflow the page
% margins by default, and it is still possible to overwrite the defaults
% using explicit options in \includegraphics[width, height, ...]{}
\setkeys{Gin}{width=\maxwidth,height=\maxheight,keepaspectratio}
% Set default figure placement to htbp
\makeatletter
\def\fps@figure{htbp}
\makeatother
% definitions for citeproc citations
\NewDocumentCommand\citeproctext{}{}
\NewDocumentCommand\citeproc{mm}{%
  \begingroup\def\citeproctext{#2}\cite{#1}\endgroup}
\makeatletter
 % allow citations to break across lines
 \let\@cite@ofmt\@firstofone
 % avoid brackets around text for \cite:
 \def\@biblabel#1{}
 \def\@cite#1#2{{#1\if@tempswa , #2\fi}}
\makeatother
\newlength{\cslhangindent}
\setlength{\cslhangindent}{1.5em}
\newlength{\csllabelwidth}
\setlength{\csllabelwidth}{3em}
\newenvironment{CSLReferences}[2] % #1 hanging-indent, #2 entry-spacing
 {\begin{list}{}{%
  \setlength{\itemindent}{0pt}
  \setlength{\leftmargin}{0pt}
  \setlength{\parsep}{0pt}
  % turn on hanging indent if param 1 is 1
  \ifodd #1
   \setlength{\leftmargin}{\cslhangindent}
   \setlength{\itemindent}{-1\cslhangindent}
  \fi
  % set entry spacing
  \setlength{\itemsep}{#2\baselineskip}}}
 {\end{list}}
\usepackage{calc}
\newcommand{\CSLBlock}[1]{\hfill\break\parbox[t]{\linewidth}{\strut\ignorespaces#1\strut}}
\newcommand{\CSLLeftMargin}[1]{\parbox[t]{\csllabelwidth}{\strut#1\strut}}
\newcommand{\CSLRightInline}[1]{\parbox[t]{\linewidth - \csllabelwidth}{\strut#1\strut}}
\newcommand{\CSLIndent}[1]{\hspace{\cslhangindent}#1}

\KOMAoption{captions}{tableheading}
\makeatletter
\@ifpackageloaded{caption}{}{\usepackage{caption}}
\AtBeginDocument{%
\ifdefined\contentsname
  \renewcommand*\contentsname{Table of contents}
\else
  \newcommand\contentsname{Table of contents}
\fi
\ifdefined\listfigurename
  \renewcommand*\listfigurename{List of Figures}
\else
  \newcommand\listfigurename{List of Figures}
\fi
\ifdefined\listtablename
  \renewcommand*\listtablename{List of Tables}
\else
  \newcommand\listtablename{List of Tables}
\fi
\ifdefined\figurename
  \renewcommand*\figurename{Figure}
\else
  \newcommand\figurename{Figure}
\fi
\ifdefined\tablename
  \renewcommand*\tablename{Table}
\else
  \newcommand\tablename{Table}
\fi
}
\@ifpackageloaded{float}{}{\usepackage{float}}
\floatstyle{ruled}
\@ifundefined{c@chapter}{\newfloat{codelisting}{h}{lop}}{\newfloat{codelisting}{h}{lop}[chapter]}
\floatname{codelisting}{Listing}
\newcommand*\listoflistings{\listof{codelisting}{List of Listings}}
\makeatother
\makeatletter
\makeatother
\makeatletter
\@ifpackageloaded{caption}{}{\usepackage{caption}}
\@ifpackageloaded{subcaption}{}{\usepackage{subcaption}}
\makeatother
\ifLuaTeX
  \usepackage{selnolig}  % disable illegal ligatures
\fi
\usepackage{bookmark}

\IfFileExists{xurl.sty}{\usepackage{xurl}}{} % add URL line breaks if available
\urlstyle{same} % disable monospaced font for URLs
\hypersetup{
  colorlinks=true,
  linkcolor={blue},
  filecolor={Maroon},
  citecolor={Blue},
  urlcolor={Blue},
  pdfcreator={LaTeX via pandoc}}

\author{}
\date{}

\begin{document}

\setstretch{1.5}
we searched through IBL dataset and selected sessions and probes that
includes channel(s) assigned to primary visual cortex. Along side the
raw LFP data we extract two other dataset for this project: trials table
containing event information such as stimulus contrast and the
onset/offset timing for each trial; channel's location detailing the
exact location of each channel and the associated brain region.

To compare the Inter-Trial Coherence (ITC) average and frequency power
across different stimulus contrast levels, a series of non-parametric
statistical tests were employed due to the non-normal distribution of
the data. The Shapiro-Wilk test was initially used to assess the
normality of the data. However, the results indicated that the data did
not follow a normal distribution. Attempts to normalize the data by
removing outliers (using the 95th percentile cutoff) were unsuccessful
in achieving normality. Consequently, non-parametric methods were
selected.

Given the non-normality and the repeated measures design of the
experiment, the Friedman test was utilized as an alternative to repeated
measures ANOVA. The Friedman test is appropriate for comparing the ITC
average and frequency power across the different conditions, as it does
not assume normal distribution and is suitable for dependent samples.

Following the Friedman test, which indicated significant differences
among conditions, post-hoc pairwise comparisons were conducted using the
Nemenyi test. The Nemenyi test is a suitable post-hoc method for the
Friedman test, allowing for the identification of specific pairs of
conditions that differed significantly in terms of ITC average and
frequency power.

This approach ensured that the statistical analysis was robust and
appropriately addressed the non-normality of the data while still
allowing for meaningful comparisons across the different stimulus
contrast levels.

Multiple Comparison for Time-Frequency ITC Estimate

To identify significant clusters in the time-frequency Inter-Trial
Coherence (ITC) estimates, we employed the one-sample permutation
cluster test as implemented in the MNE-Python library. The use of
permutation tests is particularly crucial in this context due to the
multiple comparison problem inherent in time-frequency analyses, where
statistical tests are conducted across many time points and frequency
bands. Without proper correction, this can lead to a high rate of false
positives (Maris \& Oostenveld, 2007).

Why Permutation Tests Are Needed

The multiple comparison problem arises when numerous statistical tests
are performed simultaneously, increasing the likelihood of incorrectly
rejecting at least one null hypothesis (Type I error). In the context of
time-frequency ITC estimates, testing for significance at each
time-frequency point independently would require a large number of
comparisons, leading to inflated false positive rates. Traditional
methods of controlling for multiple comparisons, such as the Bonferroni
correction, tend to be overly conservative, potentially reducing
statistical power and increasing the likelihood of Type II errors
(Nichols \& Holmes, 2002).

Permutation tests provide a non-parametric solution to this problem.
They make no assumptions about the distribution of the data, making them
particularly suitable for neurophysiological data, which often do not
follow normal distributions. Instead of relying on theoretical
distributions to determine significance, permutation tests use the data
itself to empirically construct a null distribution. This approach
allows for accurate control of the family-wise error rate (FWER) while
maintaining higher sensitivity in detecting true effects (Efron \&
Tibshirani, 1994).

How the Permutation Test Works

In our analysis, we used 1000 permutations to generate the null
distribution. The permutation test works by randomly shuffling the data
labels and recalculating the test statistic (in this case, the
t-statistic) for each permutation. This process is repeated many times
(1000 in our case), and for each permutation, the maximum t-value across
all time-frequency points is recorded. This set of maximum t-values
forms the null distribution.

The significance threshold is then defined based on this null
distribution. Specifically, the threshold is set to a value that
corresponds to the desired alpha level (e.g., 0.05), ensuring that the
probability of observing a cluster of significant time-frequency points
under the null hypothesis is appropriately controlled. By setting the
threshold to None, we allowed MNE-Python to automatically compute this
threshold based on the distribution of maximum t-values from the
permutations, ensuring that the threshold is data-driven and adapted to
the observed data distribution.

Clusters of time-frequency points that exceed this threshold are
considered statistically significant, indicating that the observed ITC
values in these clusters are unlikely to have occurred by chance. This
method not only controls for multiple comparisons but also capitalizes
on the spatial and temporal structure of the data, increasing
sensitivity to detect meaningful effects in the time-frequency domain
(Blair \& Karniski, 1993; Maris \& Oostenveld, 2007).

\subsubsection{Task detail}\label{task-detail}

In the IBL task (Figure 1), head-fixed mice had to move a visual
stimulus to the center by turning a wheel with their front paws. At the
start of each trial, the mouse was required to refrain from moving the
wheel for a quiescence period lasting between 400 and 700 milliseconds.
After this period, a visual stimulus (Gabor patch) appeared on either
the left or right side of the screen accompanied by a 100-millisecond
tone (5 kHz sine wave). If the mouse correctly moved the stimulus to the
center by turning the wheel over 35°, it received a 3 µL water reward.
Incorrect responses or failing to respond within 60 seconds resulted in
a 500-millisecond burst of white noise and a timeout (Benson et al.
2023). As shown in Figure 1c, mice typically responded quickly within 2
seconds. The stimulus is always presented for the first 1 second
regardless of response time (RT). RT is defined as the time after
stimulus when the wheel rotation exceeds the threshold.

The experiment began with 90 unbiased trials where the stimulus appeared
equally on both sides. The stimulus contrast levels were presented in a
ratio of {[}2:2:2:2:1{]} for contrasts {[}100\%, 25\%, 12.5\%, 6\%,
0\%{]}. After this initial block, trials were organized into biased
blocks where the likelihood of the stimulus appearing on one side was
fixed at 20\% for the left and 80\% for the right in ``right blocks'' or
vice versa in ``left blocks.'' These blocks consisted of 20 to 100
trials determined by a truncated geometric distribution with stimulus
contrast levels ratio identical to those in the unbiased block. In 0\%
contrast trials where no stimulus was visible, the side assignment
followed the block bias (e.g., right side for right blocks) (Benson et
al. 2023).

\begin{figure}[H]

{\centering \includegraphics[width=6.46875in,height=\textheight]{images/Untitled (10).png}

}

\caption{\textbf{a)} Example session block diagram and IBL task. Each
block of consecutive trials after 90 trials varied the probability of
the stimulus being on the right side. \textbf{B)} A timeline of the main
events and variables of the IBL task. After a quiescence period,
stimulus appears on screen alongside a go cue tone. Mice had to bring
the stimulus to the center by turning the wheel. When the wheel rotation
reaches the threshold 35 ° or after 60 s of no response, positive or
negative feedback is provided depending on the mice choice. (a) and (b)
are extracted from (Benson et al. 2023). \textbf{C)}~ Distribution of
response time (RT) with color blue and stimulus offset time with color
yellow relative to stimulus onset. Note that there is always a stimulus
presented for the first 1 second even though the mice typically answer
sooner.~}

\end{figure}%

\subsubsection{Electrophysiological
recording}\label{electrophysiological-recording}

The neural recordings were conducted using Neuropixel 1.0 probes with
384 recording channels and 960 low-impedance sites on a single shank
(Benson et al. 2023) . Neuropixel probes are advanced silicon-based
neural recording devices designed for high-density recording of neural
activity across large populations of neurons with precise spatial and
temporal resolution (Jun et al. 2017). After the recordings, electrode
tracks were reconstructed by performing serial-section 2-photon
microscopy. A region was then assigned to each recording site (and
inferred single neurons) within the Allen Common Coordinate Framework
(Benson et al. 2023).

\subsection{Preprocessing}\label{preprocessing}

\subsection{Current Source Density
(CSD)}\label{current-source-density-csd}

To remove the effects of volume conduction on the LFP data and improve
spatial resolution, we used Current Source Density (CSD) analysis. CSD
is a technique that estimates the local current flow in the brain by
calculating the second spatial derivative of the recorded potentials to
reduce the influence of distant sources. First, the Euclidean distances
between adjacent channels were computed using the channels' location
relative to the end of the probe (axial) and their location relative to
the middle of the probe (lateral). The Euclidean distance between
adjacent channels \(i\) and \(i \pm 1\) was calculated as:

\[
d_{i,i \pm 1} = \sqrt{(x_{i \pm 1} - x_i)^2 + (y_{i \pm 1} - y_i)^2}
\]

Where \(d_{i,i \pm 1}\) is the distance between channel \(i\) and its
adjacent channel \(i+1\) (next channel) or \(i-1\) (previous channel),
\(x_i\) and \(x_{i \pm 1}\) are the axial coordinates of the channels,
\(y_i\) and \(y_{i \pm 1}\) are the lateral coordinates.

Then the second spatial derivative of the LFP signals was computed as:

\[
CSD_i = \frac{V_{i+1} - V_i}{d_{i, i+1}^2} - \frac{V_i - V_{i-1}}{d_{i, i}^2}
\]

Here \(CSD_i\) represents the current source density at channel \(i\),
\(V_i\) is the voltage at channel \(i\), \(V_{i+1}\) and \(V_{i-1}\) are
the voltages at the adjacent channels \(i+1\) and \(i-1\) respectively.

In one-dimensional CSD analysis, it is typically assumed that channels
are spaced uniformly (i.e., \(d_{i i+1} = d_{i i-1}\)). However, in this
project, we accounted for non-uniform spacing to enhance accuracy and
enable the removal of noisy channels without risking the spread of
artifacts to adjacent channels. The python script for CSD computation
can be found in the ``CSD\_computation'' Jupyter notebook in the GitHub
repository.

\subsection{Time-Frequency analysis}\label{time-frequency-analysis}

For the time-frequency analysis, we chose the multitaper method. This
method is known to be well-suited for situations where specific
frequency bands are not preselected and the goal is a broad exploration
of all frequencies. Multitaper parameters were selected in a way where
frequency resolution was prioritized slightly over temporal resolution,
especially at lower frequencies. In this regard, power and phase were
calculated using MNE's multitaper function with the following
parameters: a frequency range of 2-45 Hz with a step size of 0.5 Hz for
the 2-10 Hz range and 1 Hz for frequencies above 10 Hz and a
time-bandwidth product of 3.5 with the number of cycles at each
frequency point set to half of the corresponding frequency
(\(\text{n-cycles} = \frac{f}{2}\)). These parameters were found to be
optimal for our specific data and goals. A detailed comparison of
different parameter settings can be found in the
``\emph{tf\_resolution}'' Jupyter notebook in the GitHub repository.

The

\subsection{Inter trial Phase coherence
(ITC)}\label{inter-trial-phase-coherence-itc}

Inter-trial phase coherence (ITC) was computed using MNE's built-in
function. ITC is a measure of the consistency of the phase of a signal
across different trials at a given time and frequency. Mathematically,
ITC is calculated as the magnitude of the average of normalized complex
phase values across trials. For each trial, the phase of the signal,
denoted as \(\phi(f, t)\), is extracted at each frequency \(f\) and time
point \(t\). These phase values are then represented as unit vectors on
the complex plane, i.e., \(e^{i\phi(f, t)}\).

The ITC at a particular time-frequency point is then defined as:

\[
\text{ITC}(f, t) = \left| \frac{1}{N} \sum_{n=1}^{N} e^{i\phi_n(f, t)} \right|
\]

where \(N\) is the number of trials, and \(\phi_n(f, t)\) is the phase
at frequency \(f\) and time \(t\) for the \(n\)-th trial. The resulting
ITC value ranges from 0 to 1, where 0 indicates no phase consistency
across trials, and 1 indicates perfect phase alignment across all
trials.

Introduction

Background

How does the brain efficiently process the overwhelming amount of
sensory input that it receives from a constantly changing and uncertain
environment? According to the predictive coding (PC) framework, the
brain's key solution is to actively predict incoming sensory input based
on past observations and to prioritize the processing of unpredicted
sensory information (Huang and Rao 2011). PC envisions the brain as a
prediction machine, where predictions are made through prior experience
or expectation. These predictions are compared to the actual sensory
input, and the difference, known as prediction error, is used to refine
the internal model of the environment and orient attention towards
unexpected features of the input. This allows the brain to minimize the
prediction error and allocate its finite resources more efficiently
(Vinck, Uran, and Canales-Johnson 2022).

Predictive coding is generally thought to be implemented by the cortex.
Indeed, the mammalian cortex is organized hierarchically, with lower
level areas processing more basic sensory features and higher areas
integrating this information into more complex representations (Felleman
and Van Essen 1991). Visual cortex is a prime example of this hierarchy,
with primary visual cortex (V1) processing simple visual features like
contrast and edges, and higher visual areas(e.g.~V2, V4) extracting more
complex features of stimuli like texture and shape {[}XSX{]}. The
hierarchy of cortical processing allows the brain to make predictions at
multiple levels of abstraction and compare these predictions to the
actual sensory input. according to PC the communication between
hierarchical cortical areas occurs through feedforward (FF) and feedback
(FB) connections. FB projections involve top-down propagation of
prediction from higher to lower area while FF projections involve bottom
up assembly of sensory input and prediction error from lower to higher
areas to update the internal model. FF and FB connections have distinct
laminar origins and targets within the cortex. FF connections mainly
originate from layers 3 and 5 (of lower areas) and target the layer 4
(of higher area). while, FB connections arise from layers 2 and 6 (of
higher area) and target all layers except for layer 4 (of lower
area)(Vinck, Uran, and Canales-Johnson 2022).

Prominent studies in primates visual systems (Van Kerkoerle et al. 2014)
suggest neural oscillations might play an important role in
communication between cortical regions through FF-FB connections. The
studies found that FF propagation is associated with gamma-band
oscillations, while FB propagation involves alpha-band (8--12 Hz)
oscillations. Similarly, a study on mice (Aggarwal et al. 2022) vision
revealed FF waves in the 30--50 Hz range and FB waves in the 3-6 Hz
range, where the phase of the FB oscillations modulated the amplitude of
the FF oscillations. However, this was the only study on mice, to the
best of my knowledge, that has investigated the FF-FB waves in the
visual system. Additionally, the study utilized a simple visual stimulus
that did not involve any prediction and higher-level cognitive
processes.

Despite its limited acuity, the mouse visual system offers several
advantages for studying predictive coding. The simpler and well-studied
visual processing hierarchy in mice allows for more straightforward
analysis of neural data. Genetic manipulation tools in mice also provide
unique opportunities to explore cortical micro-circuitry and manipulate
neural pathways, which is challenging in primates. Additionally, mouse
studies are more cost-effective and logistically feasible compare to
primates, making them an attractive model for large-scale explorations
of complex brain functions like predictive coding.

The International Brain Laboratory (IBL) (Benson et al. 2023) provides
an extensive open-access dataset recorded from more than 100 mice
trained to perform a perceptual decision-making task. In this task, mice
are presented with a visual stimulus of controlled contrast and are
required to move the stimulus to the center of the screen using a
steering wheel. The stimulus appears on the right or left side of the
screen, with a fixed probability for blocks of trials to create a
predictable yet changing probability that the correct response involves
a rightward or leftward movement of the wheel. This latter feature is
very interesting to study the neural implementation of predictive coding
given since leveraging the block-dependent bias in trial distribution
requires to constantly update a prior estimating how likely right or
left stimuli are before they are even displayed.

Current project

Methods

Data source and recording details

International Brain Laboratory (IBL)

We used the open-access dataset from the International Brain Laboratory
(IBL). IBL provides a comprehensive set of recordings collected from
more than 100 mice across 11 laboratories performing a standardized
perceptual decision-making task. Data are collected from 267 brain
regions by inserting 547 Neuropixel probes covering most of the left
hemisphere and spanning the forebrain, midbrain and cerebellum, as well
as the right hindbrain (Benson et al. 2023).

Task detail

In the IBL task (Figure 1), head-fixed mice had to move a visual
stimulus to the center by turning a wheel with their front paws. At the
start of each trial, the mouse was required to refrain from moving the
wheel for a quiescence period lasting between 400 and 700 milliseconds.
After this period, a visual stimulus (Gabor patch) appeared on either
the left or right side of the screen accompanied by a 100-millisecond
tone (5 kHz sine wave). If the mouse correctly moved the stimulus to the
center by turning the wheel over 35°, it received a 3 µL water reward.
Incorrect responses or failing to respond within 60 seconds resulted in
a 500-millisecond burst of white noise and a timeout (Benson et al.
2023). As shown in Figure 1c, mice typically responded quickly within 2
seconds. The stimulus is always presented for the first 1 second
regardless of response time (RT). RT is defined as the time after
stimulus when the wheel rotation exceeds the threshold.

The experiment began with 90 unbiased trials where the stimulus appeared
equally on both sides. The stimulus contrast levels were presented in a
ratio of {[}2:2:2:2:1{]} for contrasts {[}100\%, 25\%, 12.5\%, 6\%,
0\%{]}. After this initial block, trials were organized into biased
blocks where the likelihood of the stimulus appearing on one side was
fixed at 20\% for the left and 80\% for the right in ``right blocks'' or
vice versa in ``left blocks.'' These blocks consisted of 20 to 100
trials determined by a truncated geometric distribution with stimulus
contrast levels ratio identical to those in the unbiased block. Thus,
the transitions between these blocks are were largely unpredictable for
the animals. In 0\% contrast trials where no stimulus was visible, the
side assignment followed the block bias (e.g., right side for right
blocks) (Benson et al. 2023).

\begin{enumerate}
\def\labelenumi{\alph{enumi})}
\tightlist
\item
  Example session block diagram and IBL task. Each block of consecutive
  trials after 90 trials varied the probability of the stimulus being on
  the right side. B) A timeline of the main events and variables of the
  IBL task. After a quiescence period, stimulus appears on screen
  alongside a go cue tone. Mice had to bring the stimulus to the center
  by turning the wheel. When the wheel rotation reaches the threshold 35
  ° or after 60 s of no response, positive or negative feedback is
  provided depending on the mice choice. (a) and (b) are extracted from
  . C) Distribution of response time (RT) with color blue and stimulus
  offset time with color yellow relative to stimulus onset. Note that
  there is always a stimulus presented for the first 1 second even
  though the mice typically answer sooner.~
\end{enumerate}

Electrophysiological recording

The neural recordings were conducted using Neuropixel probes with 384
recording channels and 960 low-impedance sites on a single shank (Benson
et al. 2023). Neuropixel probes are advanced silicon-based neural
recording devices designed for high-density recording of neural activity
across large populations of neurons with precise spatial and temporal
resolution (Jun et al. 2017). After the recordings, electrode tracks
were reconstructed by performing serial-section 2-photon microscopy. A
region was then assigned to each recording site (and inferred single
neurons) within the Allen Common Coordinate Framework (Benson et al.
2023).

Preprocessing of electrophysiological data

Exclusion of channels and trials

Local field potential (LFP) datasets alongside their corresponding
behavioral data and channel locations were extracted for sessions that
included at least one channel in the primary visual cortex. The
destriping function of the IBL Python toolbox was applied as the first
step of preprocessing to correct for the biases induced by the
sequential acquisition of the raw voltage traces (IBL 2024). This was
followed by downsampling from 1000 Hz to 500 Hz to decrease the size of
the data files. Next, channels were excluded based on three criteria:
(i) those not located the primary visual cortex, (ii) those displaying
excessively high variance according to power spectral density, (iii) and
those with an excessively low coherence with neighboring channels.

We faced an unexpected problem with IBL LFP data due to amplifier
saturation. Indeed, Neuropixel probes (especially earlier version) have
a limited dynamic range that was frequently exceeded during the task (in
particular, when the animal licked the spout to harvest water reward).
For the analysis of spikes, this issue is less problematic as it only
prevent from detecting spikes during the saturation. However, for the
analysis of LFP, it introduces very salient artifacts and dramatically
increase inter-trial variance in power, amplitude and phase estimates,
potentially leading to erroneous conclusions. Therefore, we designed a
custom exclusion procedure tailored to capture this specific problem.
Trials were excluded based on the skewness of the absolute value of
their first-order temporal derivative (threshold set to 1.5). Indeed,
high skewness values typically reflect the presence of sudden, large
amplitude changes in an otherwise mostly flat signal. By excluding these
trials, our analyses focus on more consistent and representative
portions of the data, improving the reliability of the results. Unless
specified otherwise, all remaining trials were included in the presented
analyses (i.e., missed, incorrect and slow responses).

Common average reference

Unless otherwise specified, our electrophysiological analyses used a
common average reference scheme. The common reference was recomputed as
the mean of all channels of interest per animal (i.e., those located in
the primary visual cortex), after excluding noisy channels). This
approach was chosen to limit the influence of electrical potentials
outside of visual areas as well as the influence of non-physiological
noise.

Current Source Density (CSD)

To remove the effects of volume conduction on the LFP data and improve
spatial resolution, we used Current Source Density (CSD) analysis. CSD
is a technique that estimates the local current flow in the brain by
calculating the second spatial derivative of the recorded potentials to
reduce the influence of distant sources. First, the Euclidean distances
between adjacent channels were computed using the channels' location
relative to the end of the probe (axial) and their location relative to
the middle of the probe (lateral). The Euclidean distance between
adjacent channels \(i\) and \(i \pm 1\) was calculated as:

\[
d_{i,i \pm 1} = \sqrt{(x_{i \pm 1} - x_i)^2 + (y_{i \pm 1} - y_i)^2}
\]

Where \(d_{i,i \pm 1}\) is the distance between channel \(i\) and its
adjacent channel \(i+1\) (next channel) or \(i-1\) (previous channel),
\(x_i\) and \(x_{i \pm 1}\) are the axial coordinates of the channels,
\(y_i\) and \(y_{i \pm 1}\) are the lateral coordinates.

Then the second spatial derivative of the LFP signals was computed as:

\[
CSD_i = \frac{V_{i+1} - V_i}{d_{i, i+1}^2} - \frac{V_i - V_{i-1}}{d_{i, i}^2}
\]

Here \(CSD_i\) represents the current source density at channel \(i\),
\(V_i\) is the voltage at channel \(i\), \(V_{i+1}\) and \(V_{i-1}\) are
the voltages at the adjacent channels \(i+1\) and \(i-1\) respectively.

In one-dimensional CSD analysis, it is typically assumed that channels
are spaced uniformly (i.e., \(d_{i i+1} = d_{i i-1}\)). However, in this
project, we accounted for non-uniform spacing to enhance accuracy and
enable the removal of noisy channels without risking the spread of
artifacts to adjacent channels. The python script for CSD computation
can be found in the ``CSD\_computation'' Jupyter notebook in the GitHub
repository.

Time-Frequency power analysis

For the time-frequency analysis, we chose the multitaper method. This
method is known to be well-suited for situations where specific
frequency bands are not preselected and the goal is a broad exploration
of all frequencies. Multitaper parameters were selected in a way where
frequency resolution was prioritized slightly over temporal resolution,
especially at lower frequencies. In this regard, power and phase were
calculated using MNE's multitaper function with the following
parameters: a frequency range of 2-45 Hz with a step size of 0.5 Hz for
the 2-10 Hz range and 1 Hz for frequencies above 10 Hz and a
time-bandwidth product of 3.5 with the number of cycles at each
frequency point set to half of the corresponding frequency
(\(\text{n-cycles} = \frac{f}{2}\)). These parameters were found to be
optimal for our specific data and goals. A detailed comparison of
different parameter settings can be found in the ``tf\_resolution''
Jupyter notebook in the GitHub repository.

Baseline correction were applied to the time frequency data with
baseline defined as the interval from -0.7 to -0.5 seconds relative to
stimulus onset. For baseline correction, the percentage change method
were used which can be expressed with the following formula:

\[
\text{Corrected Power}(t,f) = (\frac{P(t,f)-\text{Baseline}(f)}{\text{Baseline}(f)}) \times 100 
\]

Where \(P(t,f)\) is the power at a specific time \((t)\) and frequency
\((f)\), and \(\text{Baseline}(f)\) is the averaged power within the
baseline interval for each frequency.

Inter trial Phase coherence (ITC)

Inter-trial phase coherence (ITC) was computed using MNE's built-in
function. ITC is a measure of the consistency of the phase of a signal
across different trials at a given time and frequency. Mathematically,
ITC is calculated as the magnitude of the average of normalized complex
phase values across trials. For each trial, the phase of the signal,
denoted as \(\phi(f, t)\), is extracted at each frequency \(f\) and time
point \(t\). These phase values are then represented as unit vectors on
the complex plane, i.e., \(e^{i\phi(f, t)}\).

The ITC at a particular time-frequency point is then defined as:

\[
\text{ITC}(f, t) = \left| \frac{1}{N} \sum_{n=1}^{N} e^{i\phi_n(f, t)} \right|
\]

where \(N\) is the number of trials, and \(\phi_n(f, t)\) is the phase
at frequency \(f\) and time \(t\) for the \(n\)-th trial. The resulting
ITC value ranges from 0 to 1, where 0 indicates no phase consistency
across trials, and 1 indicates perfect phase alignment across all
trials.

Phase-Amplitude Coupling

Phase-Amplitude Coupling (PAC) quantifies the interaction between the
phase and amplitude of two distinct frequency bands, typically involving
the phase of a low-frequency oscillation and the amplitude of a
high-frequency oscillation. In this study, PAC was computed for phase
frequencies ranging from 2 to 7 Hz and amplitude frequencies from 25 to
80 Hz using the TensorPAC Python module (Combrisson et al. 2020). The
process begins with the extraction of the instantaneous phase of the
low-frequency signal and the amplitude envelope of the high-frequency
signal carried out through Morlet wavelets. The interaction between
these signals is then evaluated to determine how the phase of slower
oscillations modulates the amplitude of faster oscillations. In this
project, the Gaussian Copula (GC) method was employed to compute PAC for
a time window spanning 500 ms before stimulus onset to 1 second after
the stimulus. Compared to other methods such as Phase Locking Value, GC
is more robust to shifts in overall signal amplitude (Combrisson et al.
2020).

The core of the GC method involves calculating the mutual information
between normalized amplitude and phase to quantify the degree to which
the phase of the low-frequency oscillation governs the amplitude of the
high-frequency oscillation. This mutual information provides a
lower-bound estimate of the PAC that is robust to overall amplitude
shifts. Mathematically, this can be expressed as: \[
gcPAC = I(a(t); \sin(\phi(t)), \cos(\phi(t)))
\]

Where \(I\) denotes the mutual information, \(a(t)\) represents the
normalized amplitude signal, and \(\phi(t)\) represents the normalized
phase signal.

After computing PAC, the values were normalized for each channel using
z-score normalization, which involves subtracting the mean and dividing
by the standard deviation. This process standardizes the PAC values and
makes them comparable across channels and subjects. Following
normalization, the values were averaged across all frequencies and for
two distinct time windows: before the stimulus (-0.5 to 0 seconds) and
after the stimulus (0 to 1 second).

Statistical Analysis

Analysis of variance (ANOVAs)

Cluster-based statistics

multiple comparison for time frequency ITC estimate

Results

Data summary

A total of 63 probes were identified in the IBL datasets, with at least
one channel assigned to the primary visual cortex (V1) (see fix X a;b
for one insertion example). From the initial dataset, 7 and 15
insertions were excluded due to over 40\% noisy channels and trials,
respectively. In the end, 41 insertions were retained, consisting of
2,262 total channels and 25,075 trials. On average, each probe was
associated with 54.83 channels in V1 (range: 2 to 118), with an average
of 532.66 trials per session (range: 276 to 1,098) (see fig X d ) .
Among the total number of channels, 212 (9.37\%) were in layer 1, 456
(20.16\%) in layer 2/3, 338 (14.94\%) in layer 4, 650 (28.74\%) in layer
5, and 606 (26.79\%) in layer 6 (see fig X c).

Figure XXX. Region of interest and recording site locations. a) Coronal
slice of the Allen Brain Atlas, highlighting the layers of the primary
visual cortex (V1) with distinct colors: yellow for layer 1, light blue
for layers 2/3, red for layer 4, blue for layer 5, and green for layer
6. The image base is extracted from the Allen Brain Atlas
(https://atlas.internationalbrainlab.org) at an Anterior-Posterior (AP)
coordinate of -3140 µm . b) Coronal slice from the Allen Brain Atlas
that shows an example of a probe insertion site in a mouse brain
(subject name: NYU-12). The black line represents the probe path,
starting in V1 and ending in the midbrain reticular nucleus
(approximately). The image is taken from the IBL online data
visualization tool (https://viz.internationalbrainlab.org). c) Pie chart
illustrating the proportional distribution of each V1 layer, using the
same color scheme as in panel (a). d) Scatter plot showing the number of
trials (range: 276-1098) and channels (range: 2-118) for the included
sessions. The mean number of channels (54.83) is indicated by a dashed
blue vertical line, and the mean number of trials (532.66) is
represented by a dashed red horizontal line.

Behavioral results

In line with previous results on whole sessions, mice performed
correctly on 80.7\% ± 5.8\% (mean ± s.d.) of the trials with reaction
time (RT) of 1.73 ± 5.7 seconds (mean ± s.d.). RT is defined as the time
interval between stimulus onset and when wheel rotation reach threshold
of 35° ; and performance is computed as a percent of correct trials over
total number of trials. As illustrated in (fig X a;b), Performance
improved and reaction times decreased on trials with higher stimulus
contrast. In 0\% contrast trials, where mice had to rely only on their
expectation and prior experience, they made correct choices in 57\% ±
8\% (mean ± s.d.).

Figure XXX. Behavior results. a) illustration of reaction time for
different stimulus contrast level using boxplot. b) Illustration of
performance (as a percent of correct trials over total number of trials)
for each contrast levels using boxplots.

Inter trial phase coherence (ITC)

The ITC analysis indicated significant phase alignment in the
low-frequency range (2-8 Hz) within the {[}0, 0.5{]} second interval
following the stimulus (see Fig. X a). To ensure that these findings
were not due to chance and to correct for multiple comparisons, we
applied the MNE one-sample cluster permutation test (refer to the
Statistical Analysis section of the Methods for more details ). The
significant clusters, marked by the black line in Fig. X a , demonstrate
that the low-frequency ITC during the 0-0.5 second period was
statistically meaningful, with a p-value of 0.001. Additionally, as
illustrated in Fig. X c, there were no significant differences in ITC
across the V1 layers in the low-frequency range.

To assess whether the observed ITC levels were influenced by the
stimulus, ITC average levels were compared for each level of stimulus
contrast. The average ITC was computed for the low-frequency range (2-8
Hz) and within the 0-0.5 second time window post-stimulus. As
illustrated in Fig. X b, an increase in stimulus contrast generally
resulted in a higher mean ITC. Interestingly, the only group of trials
that did not support this trend was trials without stimulus
(i.e.~contrast 0\%), which will be discussed in the next section.

To statistically evaluate whether the mean ITC was significantly
affected by contrast levels, further analysis was undertaken. Given that
the Shapiro-Wilk normality test did not confirm normality in the data
distribution, the Friedman test, a non-parametric alternative to
repeated measures ANOVA, was employed. The results of the Friedman test
indicated a highly significant effect of contrast level on ITC mean,
with a test statistic of 77.98 and a p-value of 4.66*10-16, indicating
that variations in ITC across different contrast levels were unlikely to
have occurred by chance. Due to the significant effect of contrast level
on ITC mean identified by the Friedman test, a post-hoc Nemenyi test was
performed to determine which specific contrast levels contributed to the
observed differences. The Nemenyi test was chosen as it is appropriate
for pairwise comparisons following a Friedman test. The post-hoc
analysis results are presented in fig X d, with p-values indicating the
significance of differences between each pair of contrast levels.

Figure XXX. Inter trial phase coherence (ITC). a) ITC average across all
subjects and cortical layers relative to stimulus onset. Significant
clusters (p = 0.001) are indicated by black lines, as determined by the
MNE one-sample cluster permutation test.b) Comparison of ITC averages
across different stimulus contrast levels using a box plot. The ITC
average was computed for the 2-8 Hz frequency range within the 0-0.5
second time window post-stimulus, aggregated across all subjects and
layers. c) Low-frequency ITC averages for each V1 layer relative to
stimulus onset. The layer-specific averages are depicted with solid
lines, and their 95\% confidence intervals are shaded around the lines
in distinct colors: yellow for layer 1, light blue for layers 2/3, red
for layer 4, blue for layer 5, and green for layer 6. d) Nemenyi
post-hoc test p-value results for each pair of contrast levels. Lower
p-values indicate significant differences between the ITC average
distributions for each pair, represented by a heatmap ranging from
yellow (low p-values) to black (high p-values).

Time frequency analysis results

Although there was substantial variability across subjects, the
time-frequency analysis of V1 revealed two notable oscillations in
relation to the visual stimulus: first, an increase in high-frequency
power within the gamma band range (20--40 Hz), and second, a concurrent
decrease in lower-frequency power within the 2--7 Hz range. The gamma
increase was more transient, while the lower frequency inhibition
persisted for a longer duration (see Figure X). As shown in Figure Xa,b,
the observed frequency band changes exhibited a similar pattern across
the different layers of V1. Statistical tests were not applied to
quantitatively evaluate the layer-specificity of these effects, as the
variability in the number of channels across layers and subjects did not
allow for such an analysis.

The averaged power of each frequency band over the first 1 second after
the stimulus was compared for trials with different contrast levels. The
Friedman test revealed a statistically significant effect of contrast
level on power modulation in both the low-frequency band (test
statistic: 20.54, p-value: 0.00039) and the gamma band (test statistic:
18.88, p-value: 0.00083), indicating that power in these bands was
significantly influenced by the stimulus contrast. However, as you can
see in Figure Xa,b, the frequency band average power across different
contrast levels is not quite observable. Additionally, the post-hoc test
results shown in Figure Xc,d indicate that the comparisons were mainly
significant only in comparison to the 100\% contrast condition.

This not readily observable difference is believed to be mainly due to
inter-subject variability, with some subjects showing significant
contrast-related modulations, while others do not. In general,
especially in sessions with high effects of contrast on power bands,
higher contrast stimuli involved greater increases in gamma and
decreases in theta band power.

Average time frequency representation. Time-frequency is averaged over
all channels in the primary visual cortex, computed using the multitaper
method. The data is baseline-corrected using the -0.7 to -0.5 s
pre-stimulus interval, with time relative to stimulus onset (marked by
dashed black vertical line). Power is shown in percent units with a
blue-to-warm colormap.

Average frequency band power across time for all layers of V1. a) low
frequency (2-7 Hz) average power for each V1 layer relative to stimulus
onset (marked by dashed black vertical line). The layer-specific
averages are depicted with solid lines, and their 95\% confidence
intervals are shaded around the lines in distinct colors: yellow for
layer 1, light blue for layers 2/3, red for layer 4, blue for layer 5,
and green for layer 6. b) similar to panel (a) but for higher frequency
band (20-40 Hz)

Affects of stimulus contrast on low and high frequency average power. a)
The box plots represents the distribution of low-frequency (2-7 Hz)
average power across different contrast levels.The power is averaged
over the first 1 second after stimulus across all channels of each
subject (N= 41). The central line in each box indicates the median power
average value, while the box edges represent the inter-quartile range
(IQR) and whiskers extend to 1.5 times the IQR. B) similar to panel (b)
but for high frequency range (20-40 hz). C) Nemenyi post-hoc test
p-value results comparing low frequency average power for pairwise
contrasts levels. Lower p-values indicate significant differences
between the low frequency average power distributions for each pair,
represented by a heatmap ranging from yellow (low p-values) to black
(high p-values). d) similar to panel (c) but for high frequency (20-40
Hz) average power

Phase amplitude coupling (PAC)

Discussion

References

\phantomsection\label{refs}
\begin{CSLReferences}{1}{0}
\bibitem[\citeproctext]{ref-aggarwal2022}
Aggarwal, Adeeti, Connor Brennan, Jennifer Luo, Helen Chung, Diego
Contreras, Max B. Kelz, and Alex Proekt. 2022. {``Visual Evoked
Feedforward{\textendash}feedback Traveling Waves Organize Neural
Activity Across the Cortical Hierarchy in Mice.''} \emph{Nature
Communications} 13 (1): 4754.
\url{https://doi.org/10.1038/s41467-022-32378-x}.

\bibitem[\citeproctext]{ref-benson2023}
Benson, Brandon, Julius Benson, Daniel Birman, Niccolò Bonacchi, Matteo
Carandini, Joana A Catarino, Gaelle A Chapuis, et al. 2023. {``A
Brain-Wide Map of Neural Activity During Complex Behaviour.''}
\emph{bioRxiv}, January, 2023.07.04.547681.
\url{https://doi.org/10.1101/2023.07.04.547681}.

\bibitem[\citeproctext]{ref-combrisson2020}
Combrisson, Etienne, Timothy Nest, Andrea Brovelli, Robin A. A. Ince,
Juan L. P. Soto, Aymeric Guillot, and Karim Jerbi. 2020. {``Tensorpac:
An Open-Source Python Toolbox for Tensor-Based Phase-Amplitude Coupling
Measurement in Electrophysiological Brain Signals.''} \emph{PLOS
Computational Biology} 16 (10): e1008302.
\url{https://doi.org/10.1371/journal.pcbi.1008302}.

\bibitem[\citeproctext]{ref-felleman1991}
Felleman, Daniel J, and David C Van Essen. 1991. {``Distributed
Hierarchical Processing in the Primate Cerebral Cortex.''}
\emph{Cerebral Cortex (New York, NY: 1991)} 1 (1): 1--47.

\bibitem[\citeproctext]{ref-huang2011}
Huang, Yanping, and Rajesh P. N. Rao. 2011. {``Predictive Coding.''}
\emph{WIREs Cognitive Science} 2 (5): 580--93.
\url{https://doi.org/10.1002/wcs.142}.

\bibitem[\citeproctext]{ref-unified2024}
IBL. 2024. {``Unified International Brain Laboratory Environment.''}
\url{https://github.com/int-brain-lab/iblenv}.

\bibitem[\citeproctext]{ref-jun2017}
Jun, James J., Nicholas A. Steinmetz, Joshua H. Siegle, Daniel J.
Denman, Marius Bauza, Brian Barbarits, Albert K. Lee, et al. 2017.
{``Fully Integrated Silicon Probes for High-Density Recording of Neural
Activity.''} \emph{Nature} 551 (7679): 232--36.
\url{https://doi.org/10.1038/nature24636}.

\bibitem[\citeproctext]{ref-vankerkoerle2014}
Van Kerkoerle, Timo, Matthew W Self, Bruno Dagnino, Marie-Alice
Gariel-Mathis, Jasper Poort, Chris Van Der Togt, and Pieter R Roelfsema.
2014. {``Alpha and Gamma Oscillations Characterize Feedback and
Feedforward Processing in Monkey Visual Cortex.''} \emph{Proceedings of
the National Academy of Sciences} 111 (40): 14332--41.

\bibitem[\citeproctext]{ref-vinck2022}
Vinck, Martin, Cem Uran, and Andrés Canales-Johnson. 2022. {``The Neural
Dynamics of Feedforward and Feedback Interactions in Predictive
Processing.''} \url{https://doi.org/10.31234/osf.io/n3afb}.

\end{CSLReferences}



\end{document}
